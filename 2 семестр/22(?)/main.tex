\documentclass[a4paper, 14pt]{article}

\usepackage[14pt]{extsizes}
\headheight=0.5cm
\headsep=0.5cm
\usepackage[T2A]{fontenc}
\usepackage[utf8]{inputenc}
\usepackage[russian]{babel}
\usepackage[left=2cm,right=2cm,
    top=1.8cm,bottom=2.5cm, marginparwidth=75pt]{geometry}
\usepackage{ifthen}
\usepackage{fancyhdr}
\usepackage{amsmath}
\pagestyle{fancy}
\fancyhf{}

\setlength{\headheight}{30pt}

\fancypagestyle{first}{
\fancyhead{}
\fancyhead[C]{\textsc{матрицы}}
\fancyhead[L]{514}
\fancyhead[R]{[\textsc{гл. xxi}}
\renewcommand{\headrulewidth}{0pt}
\fancyfoot{} 
}


\fancypagestyle{second}{
\fancyhead{}
\fancyhead[C]{\textsc{действия над матрицами. сложения матриц}}
\fancyhead[L]{\textbf{\S   4]}}
\fancyhead[R]{515}
\renewcommand{\headrulewidth}{0pt}
\fancyfoot{} 
}


\begin{document}
\thispagestyle{first}


то матрица $A$ и преобразование (1) называются \emph{вырожденными}. Это преобразование не будет взаимно однозначным.

Докажем это. Рассмотрим два возможных случая:

1) Если $a_{11}=a_{12}=a_{21}=a_{22}=0$, то при любых $x_1$ и $x_2$ будут $y_1=0, y_2 = 0$. В этом случае любая точка ($x_{1}$; $x_{2}$) плоскости $x_1Ox_2$ переходит в начало координат плоскости $y_{1}Oy_{2}$.

2) Пусть хотя бы один из коэффициентов преобразования отличен от нуля, например $a_{11} \ne 0$.

Умножая первое из уравнений (1) на $a_{21}$, второе на $a_{11}$ и производя вычитание, получим с учетом равенства (5)

\begin{center}
\begin{tabular}{l | l}
$a_{21}$ & $y_1 = a_{11}x_1 + a_{12}x_2$ \\
$a_{11}$ & $y_2 = a_{21}x_1 + a_{22}x_2$ \\
\hline
\multicolumn{2}{c}{$a_{21}y_1 - a_{11}y_2 = 0$} \\

\end{tabular}
\end{center}


\begin{flushright}
(6)
\end{flushright}

Итак, при любых $x_1$, $x_2$ для значений $y_1$ и $y_2$ получаем равенство(6), т. е. соответствующая точка плоскости $x_1Ox_2$ попадает на прямую (6) плоскости $y_1Oy_2$. Очевидно, что это отображение не является взаимно однозначным, так как каждой точке прямой (6) плоскости $y_1Oy_2$ соответствует совокупность точек плоскости $x_1Ox_2$, лежащих на прямой $y_1=a_{11}x_1 + a_{12}x_3$

В обоих случаях отображение не является взаимно однозначным.


\begin{small}
П\,р\,и\,м\,е\,р 1. Преобразование
\begin{center}
\normalsize{$y_1=2x_1 + x_2$,\hspace{5mm}$y_2=x_1 - x_2$}
\end{center}
является взаимно однозначным, так как определитель $\Delta(A)$ матрицы преобразования $A$ отличен от нуля:

\begin{center}
\normalsize{$\Delta(A) = \begin{vmatrix}
2 & \hspace{3mm}1\\
1 & -1
\end{vmatrix} = -3$.}
\end{center}
Обратное преобразование будет
\begin{center}
    \normalsize{$x_1 = \frac{1}{3}y_1 + \frac{1}{3}y_2$,\hspace{5mm}$x_2 = \frac{1}{3}y_1 - \frac{2}{3}y_2$}
\end{center}
Матрица обратного преобразования, в соответствии с формулой (4) будет
\begin{center}
\Large{
$A^{-1} = \begin{Vmatrix}
\frac{1}{3} & \hspace{5mm}\frac{1}{3}\\
\frac{1}{3} & -\frac{2}{3}
\end{Vmatrix}$.}
\end{center}

П\,р\,и\,м\,е\,р 2. Линейное преобразование
\begin{center}
\normalsize{$y_1=x_1 + 2x_2$,\hspace{5mm}$y_2=2x_1 + 4x_2$}
\end{center}
является вырожденным, так как определитель матрицы преобразования
\begin{center}
\normalsize{$\Delta(A) = \begin{vmatrix}
1 & 2\\
2 & 4
\end{vmatrix} = 0$.}
\end{center}
Это преобразование переводит все точки плоскости $(x_1, x_2)$ в прямую $y_2 - 2y_1 = 0$ плоскости $(y_1, y_2).$
% \maketitle
\end{small}
\newpage
\thispagestyle{second}
\begin{center}
\textbf{\large{\textsection{4 Действия над матрицами. Сложения матриц}}}%
\end{center}

О\,п\,р\,е\,д\,е\,л\,е\,н\,и\,е 1. \emph{Суммой} двух матриц $\begin{Vmatrix}a_{ij}\end{Vmatrix}$ и $\begin{Vmatrix}b_{ij}\end{Vmatrix}$ c одинаковым количеством строк и одинаковым количеством столбцов называется матрица $\begin{Vmatrix}c_{ij}\end{Vmatrix}$, у которой элементом $c_{ij}$ является сумма $a_{ij} + b_{ij}$ соответствующих элементов матриц $\begin{Vmatrix}a_{ij}\end{Vmatrix}$ и $\begin{Vmatrix}b_{ij}\end{Vmatrix}$, т. e.
\begin{equation}
        \begin{Vmatrix}a_{ij}\end{Vmatrix} + \begin{Vmatrix}b_{ij}\end{Vmatrix} = \begin{Vmatrix}c_{ij}\end{Vmatrix}, \hbox{если}
\end{equation}
\begin{equation}
    a_{ij} + b_{ij} = c_{ij}\hspace{5mm}(i = 1, 2, ..., m;\hspace{2mm}j = 1, 2, ..., n).
\end{equation}

П\,р\,и\,м\,е\,р 1. $\begin{Vmatrix}
a_{11} & a_{12} \\
a_{21} & a_{22}
\end{Vmatrix}$ + 
$\begin{Vmatrix}
b_{11} & a_{12} \\
a_{21} & a_{22}
\end{Vmatrix}$ =
$\begin{Vmatrix}
a_{11} + b_{11} & a_{12} + b_{12} \\
a_{21} + b_{21} & a_{22} + b_{22}
\end{Vmatrix}$.\\*

Аналогичным образом определяется \emph{разность} двух матриц.
Целесообразность такого определения суммы двух матриц,
в частности, следует из представления вектора как столбцевой
матрицы:

У\,м\,н\,о\,ж\,е\,н\,и\,е\hspace{5mm}м\,а\,т\,р\,и\,ц\,ы\hspace{5mm}н\,a\hspace{5mm}ч\,и\,с\,л\,о. Чтобы умножить умножить матрицу на число $\lambda$, нужно умножить на это число каждый элемент матрицы: 
\begin{equation}
    \lambda\begin{Vmatrix}a_{ij}\end{Vmatrix} = \begin{Vmatrix}\lambda a_{ij}\end{Vmatrix}
\end{equation}

Если $\lambda$ целое, то формула (3) получается как следствие правила сложения матриц.

П\,р\,и\,м\,е\,р 2. $\lambda\begin{Vmatrix}
a_{11} & a_{12} \\
a_{21} & a_{22}
\end{Vmatrix}$ = 
$\begin{Vmatrix}
\lambda a_{11} & \lambda a_{12} \\
\lambda a_{21} & \lambda a_{22}
\end{Vmatrix}$.\\*

П\,р\,о\,и\,з\,в\,е\,д\,е\,н\,и\,е\hspace{5mm}д\,в\,у\,x\hspace{5mm}м\,а\,т\,р\,и\,ц. Пусть имеем линейное преобразование плоскости $x_1Ox_2$ на плоскость $y_1Oy_2$:
\begin{equation}
    y_1 = a_{11}x_1 + a_{12}x_2,\hspace{5mm}y_2 = a_{21}x_1 + a_{22}x_2
\end{equation}
с матрицей преобразования
\begin{equation}
    A = \begin{Vmatrix}
a_{11} & a_{12} \\
a_{21} & a_{22}
\end{Vmatrix}.
\end{equation}

Пусть, далее, произведено линейное преобразование плоскости $y_1Oy_2$ на плоскость $z_1Oz_2$:
\begin{equation}
    z_1 = b_{11}y_1 + b_{12}y_2,\hspace{5mm}z_2 = b_{21}y_1 + b_{22}y_2
\end{equation}
c матрицей преобразования
\begin{equation}
    B = \begin{Vmatrix}
b_{11} & b_{12} \\
b_{21} & b_{22}
\end{Vmatrix}.
\end{equation}

Требуется определить матрицу преобразования плоскости $x_1Ox_2$ на плоскость $z_1Oz_2$. Подставляя выражение (4) в равенства (6), получаем
$$z_1 = b_{11}(a_{11}x_1 + a_{12}x_2) + b_{12}(a_{12}x_1 + a_{22}x_2),$$
$$z_2 = b_{21}(a_{11}x_1 + a_{12}x_2) + b_{22}(a_{12}x_1 + a_{22}x_2),$$
\end{document}



