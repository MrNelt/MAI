\documentclass[12pt]{article}

\usepackage[utf8]{inputenc}
\usepackage[russian]{babel}

\renewcommand{\emph}[1]{\textbf{#1}}

\begin{document}
\subsubsection*{Закончилась квалификация перед Гран При Кореи\\
\footnotesize{23.10.2010 16:00}}

Мы ожидали настоящей битвы за поул-позишн после утренней тренировки в~субботу, и~квалификация первого Гран При Кореи нас не~разочаровала, потому что пара пилотов \emph{Red Bull} Себастьян Феттель и~Марк Уэббер опередили \emph{Ferrari} Фернандо Алонсо, буквально в самый последний момент заняв первый ряд.

Алонсо установил темп в~своей первой попытке, а~результат второй попытки 1'35,766'', казалось, решил вопрос в~пользу \emph{Ferrari}. Но затем Феттель, выжав из своего \emph{RB6} 1'35,585'', выхватил поул-позишн, а~через несколько секунд своим третьим кругом с~результатом 1'35,659'' Уэббер отправил испанца еще ниже.

\emph{McLaren}-у же не~хватило скорости, чтобы поучаствовать в~этой неистовой битве. После ошибки, отправившей его на пит-лейн, Льюис Гамильтон довольствовался четвертым местом \emph{McLaren}-а, с~результатом 1'36,062'' заполнив второй ряд.

Нико Росберг подтвердил восстановление формы \emph{Mercedes GP}, заняв пятое место с~результатом 1'36,535'' и~разделив третий ряд с~\emph{Ferrari} Фелипе Массы, который оказался там в~силу своей первой попытки с~результатом 1'36,571''. Ни~разу в~ходе квалификации Дженсон Баттон не~выглядел уверенным и~оказался седьмым на другом \emph{McLaren}-е с~результатом 1'36,731'', разделив четвертый ряд с~Робертом Кубицей, который прошел круг на своем \emph{Renault} за 1 м 36,824 с.

Михаэль Шумахер на втором \emph{Mercedes}-е был девятым с~результатом 1'36,950'', а~Рубенс Баррикелло на Williams-е 10-м~--- 1'36,998''.
\end{document}
