\documentclass{article}
\usepackage[utf8]{inputenc}
\usepackage{indentfirst}
\usepackage[english,russian]{babel}
\usepackage{geometry}
\newcommand{\RNumb}[1]{\uppercase\expandafter{\romannumeral #1\relax}}% новая команда \RNumb для вывода римских цифр

\begin{document}
\pagestyle{empty}
\section*{\centering\small{Г\,л\,а\,в\,а\, $\RNumb{2}$}}
\section*{\centering\large\textbf{ИНТЕРПОЛЯЦИЯ И СМЕЖНЫЕ ВОПРОСЫ}}
\Large Задача приближения функций возникает при решении многих задач, а иногда как самостоятельная. Настоящая глава посвящена частному, но довольно распространенному способу приближения функций путем интерполяции их значений; интерполяция является также важным вспомогательным аппаратом при решении других задачи численного анализа: численного интегрирования и дифференцирования, решения дифференциальных уравнений  и др. Прежде ем переходить непосредственно к интерполяции, напомним некоторые определения и в $\S 1$ обсудим  различные постановки задачи приближения функций.
\par\normalsize Множество $M$ называется $\textit{линейным}$, если в нем определены операции сложения и умножения на числа (действительные или комплексные), не выходящие за пределы $M$ и удовлетворяющие условиям:
\par 1) сложение ассоциативно: $(x + y) + z = x + (y + z)$,
\par 2) коммутативно $x + y = y + x$,
\par 3) существует нулевой элемент 0 для которого $x + 0 = x$ при любом $x \in M$,
\par 4) 0$\cdot x = 0$ при любом $x \in M$.
\par 5) $(\alpha + \beta)x = \alpha x + \beta x$,
\par 6) $\alpha (x + y) = \alpha x + \beta x$
\par 7) $\alpha (\beta x) = (\alpha \beta )x$
\par 8) $1 \cdot x = x$
\par В линейном множестве можно ввести понятие линейной зависимости и линейной независимости элементов. Система элементов $x_1 , \dots , x_n$ линейного множества $M$ называется $\textit{линейно зависимой}$, если существуют $c_1 , \dots , c_n$, не равных одновременно нулю, такие, что $$c_1 x_1 + \dots + c_n x_n = 0.$$ В противном случае систему называют $\textit{линейно независимой}$.
\par $\textit{Линейным подпространством}$ называется подмножество $H$ линейного множества, для которого из условия $x,y \in H$ следует  $\alpha x + \beta y \in H$ при любых $\alpha$ и $\beta$.
\par Пространство $R$ называется $\textit{метрическим}$, если  для любых двух элементов определено расстояние $\rho (x,y)$, удовлетворяющее условиям:
\par 1) $\rho (x,y) \geq 0$б причем $\rho (x,y) = 0$ тогда и только тогда, когда  $x=y$,
\par 2) $\rho (x,y) = \rho (y,x)$,
\par 3) $\rho (x,y) \leq \rho(x,z) + \rho (z,y)$ для любых $x,y,z \in R$.
\newpage

\S 1 \hfill \smallПОСТАНОВКА ЗАДАЧИ ПРИБЛИЖЕНИЯ ФУНКЦИИ \hfill 31
\newline \newline

\par Множество $R$ называют $\textit{линейным нормированным пространством}t$, если a) оно линейно и б) каждому элементу $f \in R$ поставлено в соответствие действительное число $\parallel f \parallel$, называемое $\textit{нормой}$ $f$ и удовлетворяющее условиям: 
\par 1) $\parallel f \parallel \geq 0$, причем $\parallel f \parallel = 0$ тогда и только тогда, когда $f=0$,
\par 2) $\parallel \alpha f \parallel = \mid \alpha \mid  \parallel f  \parallel$ для любого комплексного $\alpha$,
\par 3) $\parallel f_1 + f_2 \leq \parallel f_1 \parallel + \parallel f_2 \parallel$.
\par Очевидно, линейное нормированное пространство является одновременно метрическим расстоянием $$ \rho (f_1,f_2)= \parallel f_1 - f_2 \parallel .$$
\par Линейное нормированное пространство называется $\textit{строго нормированным}$, если равенство $$\parallel f_1 = f_2 \parallel = \parallel f_1 \parallel + \parallel f_2 \parallel$$возможно тогда и только тогда, когда $f_2 = \alpha f_1, \alpha \geq 0$.
\par Говорят, что в линейном множестве $R$ определено скалярное произведение, если каждой упорядоченной паре элементов $f_1 , f_2 \in R$ поставлено в соответствие некоторое комплексное число $(f_1,f_2$ и при этом выполняются соотношения:
\par 1. $(f_1,f_2)= \overline{(f_2,f_1)}.$
\par 2. Для любых $f_1,f_2,f_3 \in R$ и комплексных $\alpha_1,\alpha_2$ имеет место равенство $$(\alpha_1 f_1 + \alpha_2 f_2, f_3)=\alpha_1 (f_1,f_3) + \alpha_2 (f_2,f_3).$$
\par 3. $(f,f) \geq 0$ и $(f,f)=0$ только при $f=0.$
\par Из этих свойств скалярного произведения вытекает ряд других его свойств:
\par 4. $(f_3 , \alpha_1 f_1 + \alpha_2 f_2)=\overline{\alpha_1}(f_3,f_1)+\overline{\alpha_2}(f_3,f_2).$
\par Для любых $f_1, f_2 \in R$ имеют место неравенства:
\par $\mid (f_1,f_2)\mid^2 \leq (f_1,f_1)(f_2,f_2).$
\par $\parallel f_1 + f_2 \parallel \leq \parallel f_1 \parallel + \parallel f_2 \parallel$, где $\parallel f \parallel = \sqrt{(f,f)}.$
\par 7. Знак равенства в п.6 имеет место лишь при $f_2 = \alpha f_1, \alpha \geq 0$.
\par Свойство 6 означает, что линейное множество $R$ со скалярным произведением является линейным нормированным и, следовательно, метрическим пространством с $$\rho (f_1,f_2)=\parallel f_1 - f_2 \parallel = \sqrt{(f_1 - f_2 f_1 - f_2)}.$$Свойство 7 означает, что оно является строго нормированным.
\section*{\centering$\mathsection{1.}\textbf{\large Постановка задачи приближения функций}$}
\par Задача приближения функции возникает в различных ситуациях, часть их которых будет рассмотрена далее. Многообразие методов, предлагаемых для ее решения, столько велико, что иногда возникает следующий вопрос. Может быть, наличие большого количества различных способов приближения объясняется просто отсутствием научного подхода к постановке и решению проблемы; если бы такой подход был, то, может быть, удалось бы предложить один оптимальный способ приближения, пригодный во всех случаях? Такой вопрос возникает и при рассмотрении других разделов численного анализа. Сколько бы ни было заманчиво разработать единый подход к решению всех задач,
\end{document}

