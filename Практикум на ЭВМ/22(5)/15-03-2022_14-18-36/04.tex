\documentclass[12pt]{article}

\usepackage[utf8]{inputenc}
\usepackage[russian]{babel}

\usepackage{amsmath}
\usepackage{amsfonts}
\usepackage{amssymb}
\usepackage{wasysym}

\usepackage{csquotes}
\usepackage{textcomp}

\begin{document}

\section{Простые формулы}

Конек \TeX-а~--- это формулы, как встраиваемые типа $|\sin x| \leqslant 1$, так и выключенные вроде

$$e^x = \sum_{n = 0}^\infty \frac{x^n}{n!} = 1 + \sum_{n = 1}^\infty \frac{x^n}{n!}.$$

Команда \textbackslash frac хитроумная: она выбирает размер шрифта в зависимости от того, является формула строчной ($\frac{a}{b}$) или выключенной: $$\frac{a}{b}.$$ Команда \textbackslash dfrac сохраняет размер: $\dfrac{a}{b}$. Впрочем, это может показаться неэстетичным, ведь межстрочный интервал меняется по ходу абзаца! (Это предложение написано специально, чтобы абзац был достаточно длинным и было можно визуально оценить изменение межстрочного интервала.)

Стрелки и прочая диакритика никогда не была проблемой \TeX-а: $\vec a \times \vec b$, $\hat c \cdot \tilde d$, $\grave e$, $\ddot f$, $\bar g$, $\check \imath$, $\acute \jmath$, $\dot k$, $\breve l$.


\section{Выбор шрифта}

В формулах выбор шрифтов еще богаче, чем в обычном тексте:
\begin{description}
\item[прямой] $\mathrm a + b$;
\item[полужирный] $\mathbf a + b$;
\item[моноширинный] $\mathtt a + b$;
\item[рубленый] $\mathsf a + b$;
\item[каллиграфический] $\mathcal A + b$ (существует только для заглавных букв);
\item[готический] $\mathfrak a + B$;
\item[ажурный] $a \in \mathbb R^2$ (тоже только для заглавных букв);
\item[греческий] $\alpha \beta \Gamma \Delta$.
\end{description}


\section{Сложные формулы}

Кроме обычных скобок в \TeX-е есть скобки переменного размера (и, само собой, матрицы):
$$\left\| \begin{array}{cc}%
	a & b \\%
	c & d \\%
\end{array}\right\| = ad - cb,$$
даже \emph{очень} переменного размера:
$$b{j,2} = \left\{ \begin{array}{l}%
	\frac12 t^2 \\
	-t^2 + t + \frac12 \\
	\frac12 (1 - t)^2
\end{array}\right..$$

(Формула абзацем выше~--- однородный квадратичный B-сплайн.)

Иногда нужно записать очень большую матрицу:
$$\left(\begin{array}{cccc}
	a_{11} & a_{12} & \cdots & a_{1n} \\
	a_{21} & a_{22} & \cdots & a_{2n} \\
	\vdots & \vdots & \ddots & \vdots \\
	a_{m1} & a_{m2} & \cdots & a_{mn} \\
\end{array}\right).$$

Формула в рамке~--- не~что-нибудь, а, на минутку, формула Остроградского:
$$\boxed{\iiint_V (\nabla \cdot \mathbf F)\;dV = \oiint_S (\mathbf F \cdot \mathbf n)\;dS}.$$

\section{Нумерация формул}

Сами по себе выключенные формулы не нумеруются, но окружение \enquote{equation} автоматически их нумерует.

Формула Грина:
\begin{equation}
\label{fG}
\int_{\partial D} P\;dx + Q\;dy = \int_D \left( \frac{\partial Q}{\partial x} - \frac{\partial P}{\partial x} \right)\;dx\;dy.
\end{equation}

Формула Кельвина-Стокса:
\begin{equation}
\label{fKS}
\int_\Sigma \mathrm{rot}\,\mathbf F\;d\mathbf \Sigma = \int_{\partial\Sigma} \mathbf F\;d\mathbf r.
\end{equation}

Формулы (\ref{fG}) и (\ref{fKS}) являются частными случаями формулы Стокса:
\begin{equation}
\int_\sigma d\omega = \int_{\partial \sigma} \omega.
\end{equation}

\section{Многострочные формулы}

\begin{multline}
\label{fO}
\iint_{\partial V} P\;dy\;dz + Q\;dz\;dx + R\;dx\;dy = \\
= \iiint_V \left(\frac{\partial P}{\partial x} + \frac{\partial Q}{\partial y} + \frac{\partial R}{\partial z} \right)\;dx\;dy\;dz.
\end{multline}

Формула (\ref{fO})~--- вариант формулы Остроградского.

\end{document}
