\documentclass[10pt, a5paper]{book}

\usepackage[OT1]{fontenc}
\usepackage[utf8]{inputenc}
\usepackage[english, russian]{babel}
\usepackage{soulutf8}

\usepackage[left=1cm,right=1.5cm,top=1cm,bottom=0.5cm,bindingoffset=0cm]{geometry}
\usepackage{setspace}
\linespread{0.9}
\let\emph\textit

\usepackage{amsmath, amssymb}
\usepackage{wasysym}

\begin{document}

\markboth{\small{\textsc{дифференциальные уравнения высших порядков \qquad \small{[\,гл. IV}}}}
{\textsc{\S \ 1]\hspace{3.5cm}теорема существования}}

\setcounter{page}{132}

\noindentДопустим, что для члена $y_i{}^{(m-1)}(x)-y_i{}^{(m-2)}(x)$ мы получим оценку\linebreak \\
$\vert \, y_i{}^{(m-1)}(x)-y_i{}^{m-2}(x) \, \vert\leqslant$
$$
\leqslant M(nK)^{m-2} \frac{\vert x-x_0\vert}{(m-1)!}^{m-1} \qquad (i=1, \ 2, \, \dots \, , \ n); \eqno(8_{m-1})
$$
покажем, что аналогичная оценка справедлива для следующего члена:\linebreak \\
$\vert \, y_i{}^{(m)}(x)-y_i{}^{(m-1)}(x) \, \vert =$ \\

\noindent$=\vert \int\limits_{x_0}^x \, [f_i \, (x, \, y_1{}^{(m-1)}, \, \dots \, , \, y_n{}^{(m-1)})-f_i \, (x, \, y_1{}^{(m-2)}, \, \dots \, , \, y_n{}^{m-2})] \, dx \, \vert\leqslant $ \\

\noindent$\leqslant\vert \int\limits_{x_0}^x \,\vert f_i \, (x, \, y_1{}^{(m-1)}, \,  \dots \, , \, y_n{}^{(m-1)})-f_i \, (x, y_1{}^{(m-2)}, \, \dots \, , \, y_n{}^{(m-2)}) \, \vert \ dx \, \vert\leqslant$ \\

$\hspace{3.5cm}\leqslant K \, \vert \int\limits_{x_0}^x \sum\limits_{l=1}^n \, \vert \, y_l{}^{(m-1)}-y_l{}^{(m-2)} \, \vert \, dx \,\vert \, \leqslant $ 

$$
\hspace{1.65cm}\leqslant M(nK)^{m-1} \, \left\vert \, \int \limits_{x_0}^x \frac{\vert \, x-x_0 \, \vert ^{m-1}}{(m-1)!}dx \, \right\vert = 
$$ 

$$= M \, (nK)^{m-1} \frac{\vert \, x-x_0 \, \vert}{m!}^m. \eqno(8_m) $$



\noindent Таким образом мы доказали, что оценка $(8_m)$ справедлива для всякого \linebreak
натурального m. Замечая далее, что $\vert \, x-x_0 \, \vert\leqslant h$, мы видим, что все \linebreak
члены рядов (8), начиная со второго, соответственно не больше по \linebreak
абсолютной величине, чем члены \so{знакоположитeльного числового ряда}
$$\sum\limits_{m=1}^{\infty} M(nK)^{m-1}\frac{h^m}{m!}.$$
Этот последний ряд, как легко проверить, сходится; следовательно,\linebreak
ряды (8) сходятся равномерно для значений $x$ в отрезке $x_0-h\leqslant x\leqslant$ \linebreak
$\leqslant x_0+h$; так как их члены суть непрерывные функции, то и суммы\linebreak
их будут фукнциями непрерывными. Обозначим их через $Y_i(x)$\linebreak
$\qquad (i=1, \, 2, \, \dots \, , \ n)$. \\
\indentМы имеем:
$$Y_i(x)=y_i{}^{0}+\sum\limits_{l=1}^{\infty}( \, y_i{}^{(l)}-y_i{}^{(l-1)})=\lim_{m\to\infty}y_i{}^{(m)}(x).$$
Докажем, что \emph{функции} \\
$$Y_1 \,(x), \, Y_2 \,(x), \ \dots \, , \, Y_n \,(x)$$
\emph{дают искомую систему решений системы дифференциальных урав-\linebreak нений} (4).
\newpage
По самому определению $y_i{}^{(m)}(x)$ \,[\,см.\,($7_m$)] мы имеем $y_i{}^{(m)}(x_0)=y_i{}^{0},$\linebreak
следовательно, \\
$$\lim_{m\to\infty} y_i{}^{(m)}(x_0)=Y_i(x_0)=y_i{}^{0},$$
т. е. предельные \emph{функции $Y_i(x)$ удовлетворяют начальным условиям.}\linebreak
\indentДокажем, что эти функции удовлетворяют системе (4). В силу \linebreak равенства $(7_m)$ мы можем написать: \\

\noindent$y_i{}^{(m)}(x)=y_i{}^{0}+\int\limits_{x_0}^{x} \,\{\,f_i\,[\,x, \, y_1{}^{(m-1)}(x), \, \dots \, , \, y_n{}^{(m-1)}\,(x)]\,-$ \\

$\hspace{1.7cm}-f_i\,[\,x, \, Y_1(x), \, \dots \, , \, Y_n\,(x)]\;\} \;dx \,+ $\\

$\hspace{1.7cm}+\int\limits_{x_0}^{x}f_i\,[\,x,\, Y_1(x), \,  \dots \, , \, Y_n(x)] \; dx \qquad (i=1, \ 2, \, \dots \, , \ n). \hspace{1.2cm}(9)$ 

\noindentОценим абсолютную величину первого интеграла: \\

$\hspace{0.3cm}\vert\int\limits_{x_0}^{x} \,\{f_i\,(x, \, y_1{}^{(m-1)}, \, \dots \, , \, y_n{}^{(m-1)})-f_i\,(x, \, Y_1, \, \dots \, , \, Y_n)\,\}\, dx \, \vert \leqslant $ \\

$\leqslant \vert\int\limits_{x_0}^{x}\vert \, f_i\,(x,\, y_1{}^{(m-1)},\,  \dots \, , \, y_n{}^{(m-1)})-f_i\,(x,\, Y_1, \, \dots \, , \, Y_n) \,\vert \, dx \, \vert \leqslant $ \\

\noindent $\hspace{0.1cm}\leqslant K\vert\int\limits_{x_0}^{x}\{ \,\vert \, y_1{}^{(m-1)}-Y_1\,\vert+  \dots \, +\vert y_n{}^{(m-1)\,}-Y_n\vert\,\}\, dx\, \vert.\qquad\qquad\qquad\qquad\qquad(9')$ \linebreak
(Последнее неравенство есть следствие условий Липшица). Так как\linebreak
функции $y_i{}^{(m-1)}\,(x) \ (m=1,\, 2, \, \dots \, )$ сходятся в интервале $(x_0-h, x_0+h)$\linebreak
равномерно к $Y_i\,(x) \ (i=1, 2,  \dots ,\, n)$, то для любого наперед за-\linebreak
данного $\varepsilon$ можно найти такое $N$, что для $m-1>N$ и для всякого\linebreak
значения $x$ в рассматриваемом интервале выполняются неравенства: \\

$\hspace{1.6cm}\vert \, y_i{}^{(m-1)}(x)-Y_i\,(x)\,\vert <\dfrac{\varepsilon}{nKh} \hspace{3.1cm} (i=1, \ 2, \, \dots \, , \ n),$ \\

и тогда для первого интеграла в формуле (9) получается в силу не-\linebreak
равенства (9') оценка при $\vert \, x-x_0 \,\vert \leqslant h:$
$$\vert\int\limits_{x_0}^{x}\{f_i\,(x,\, y_1{}^{(m-1)} ,\, \dots \, , \, y_n{}^{(m-1)})-f_i\,(x,\, Y_1, \, \dots \, , \, Y_n)\,\} \ dx \ \vert <\frac{\varepsilon}{nKh}hnK=\varepsilon.$$ \\
\indentСледовательно, при $m\to\infty$ предел этого интеграла равен нулю.\linebreak
С другой стороны, по доказанному, $\lim\limits_{m\to\infty} y_i{}^{m}(x)=Y_i(x),$ и равенства (9) \\

\noindent дают: \\

$\hspace{0.5cm}Y_i(x)=y_i{}^{0}+\int\limits_{x_0}^xf\,(x, \, Y_1, \, \dots \, , \, Y_n)\,dx \hspace{3.1cm} (i=1,\,2,\, \dots \, , \, n);$
\end{document}
