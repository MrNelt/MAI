\documentclass[a5paper,10pt]{book}
\usepackage[T1]{fontenc}
\usepackage[utf8]{inputenc}
\usepackage[english,russian]{babel}
\usepackage{amsmath}
\usepackage{setspace}

\usepackage[left=1cm,right=1.8cm,top=1.2cm,bottom=3cm,bindingoffset=0cm]{geometry}
\linespread{0.95}
\setcounter{page}{138}
\setlength{\textwidth}{330pt}
\setlength{\textheight}{510pt}
\:

\begin{document}
\markboth{\small{\qquad\textsc{диференциальные уравнения высших порядков\hspace{0.4cm} \small{[гл. IV}}}}
{\small{\textsc{{\S ~1]}\hspace{3cm}теорема существования}}}
\noindent
обыкновенно получаем другие произвольные постоянные; однако если их число равно n, то мы при выполнении некоторых условий сумеем из формулы (14') или (15) получить (в некоторой области) любое частное решение, т. е. решение удовлетворяющее начальным данным Коши; при выполнении упомянутых условий мы также будем называть формулу (14') общим решением, а формулу (15) общим интегралом уравнения (1) или (1'), причем постоянные $C_{1}$, $C_{2}$, ...,$C_{n}$ уже не являются непременно начальными значениями $y$, $y'$, ..., $y^{(n-1)}$.
\\ \indent
Покажем, как р\,е\,ш\,и\,т\,ь\, з\,а\,д\,а\,ч\,у\, К\,о\,ш\,и\,, е\,с\,л\,и\, и\,з\,в\,е\,с\,т\,н\,о\, о\,б\,щ\,е\,е\, р\,е\,ш\,е\,н\,и\,е\, (14'). Из соотношения (14') и тех, которые получаются из него диференцированием по $x$, подставляя в них вместо $x$  начальное значение $x_{0}$, а вместо $y$, $y'$, ..., $y^{(n-1)}$ их начальные значения мы получим равенства:
$$
\left.
	\begin{array} {ccc}
\varphi(x_{0}, C_{1}, C_{2}, ..., C_{n})=y_{0}, \\
\varphi'(x_{0}, C_{1}, C_{2}, ..., C_{n})=y_{0}', \\
.  ~~  .  ~~  .  ~~  .  ~~  .  ~~  .  ~~  .  ~~  .  ~~  .  ~~  .  ~~  .  ~~  .  ~~  .\\
.  ~~  .  ~~  .  ~~  .  ~~  .  ~~  .  ~~  .  ~~  .  ~~  .  ~~  .~~.  ~~  .  ~~  .\\
\varphi^{(n-1)}(x_{0}, C_{1}, C_{2}, ..., C_{n})=y_{0}^{(n-1)}; 
	\end{array}
\right\} \eqno{(16)}\
$$
\noindent
рассматривая равенства (16), как  \textit{n}  уравнений с \textit{n} неизвестными $C_1$, $C_2$, ..., $C_n$, мы получим, вообще говоря, числовые значения $C_{1}, C_{2}, ..., C_{n},$ соответствующие тому частному решению, которое отвечает данным начальным условия (2). Точно так же, если дан общий интеграл (15), то, подставляя в него вместо $y$ решение (14') мы получим тождество; диференцируем его по $x$, помня, что $y$ является функцией $x$, и подставляем в полученные равенства начальные значения (2), получаем:
$$
\left.
\begin{aligned}
		\Phi(x_0, y_0, C_1, C_2, ..., C_n)=0, & \\
		\left(\frac{\partial \Phi}{\partial x}\right)_0 +
		\left(\frac{\partial \Phi}{\partial y}\right)_0 y_{0}'=0,  \\
		\left(\frac{\partial^2 \Phi}{\partial x^2}\right)_{0} +
	 2 \left(\frac{\partial^2 \Phi}{\partial x \partial y}\right)_0 y_{0}'+
		\left(\frac{\partial^2 \Phi}{\partial y^2}\right)_0 y_{0}^{,2} + 	
		\left(\frac{\partial \Phi}{\partial y}\right)_0 y_{0}^n=0,  \\	
.  ~~  .  ~~  .  ~~  .  ~~  .  ~~  .  ~~  .  ~~  .  ~~  .  ~~  .  ~~  .  ~~  .  ~~  ..  ~~  .  ~~  .  ~~  .  ~~  .  ~~  .  ~~  .  ~~  .  ~~  .  ~~  .  ~~  .  \\
.  ~~  .  ~~  .  ~~  .  ~~  .  ~~  .  ~~  .  ~~  .  ~~  .  ~~  .  ~~  .  ~~  .  ~~  ..  ~~  .  ~~  .  ~~  .  ~~  .  ~~  .  ~~  .  ~~  .  ~~  .  ~~  .  ~~  .   \\
		\left(\frac{\partial^{n-1} \Phi}{\partial x^{n-1}}\right)_{0} +
		 . . . + \left(\frac{\partial \Phi}{\partial y}\right)_{0} y_{0}^{n-1}=0 &
\end{aligned} 
\right\} \eqno{(16')}\
$$

\noindent
(символ $( ~ )_{0}$ указывает, что в данном выражении вместо $x$ и $y$ следует подставить $x_{0}$ и $y_{0}$). \\ \indent
Мы опять получаем \textit{n} уравнений для определения \textit{n} неизвестных $C_1$, $C_2$, ..., $C_n $, т. е. мы и в этом случае можем, вообще говоря, решить задачу Коши. \\ \indent
П\,р\,и\,м\,е\,ч\,а\,н\,и\,е\, 1. Разрешение системы (16) и (16') относительно $C_1$, $C_2$, ..., $C_n$ заведомо возможно лишь для тех начальных значений, при которых выполняются условия существования неявных функций, т. е. вблизи такой системы значений $ \overline x_{0}, \overline y_{0}, \overline y_{0}', ..., \overline y_{0}^{(n)}, \overline C_{1}, \overline C_{2}, . . . , \overline C_{n}, $ которые удовлетворяют системе (16) или (16') и для которых якобиан от левых частей соответствующих уравнений по $ C_{1}, C_{2}, . . . , C_{n} $ не обращается в нуль. Если этот якобиан тождественно равен нулю, то определение $ C_1$, $C_2$, . . . , $C_n $, т. е. решение задачи Коши невозможно для произвольных начальных значений $ y_0, y_{0}', . . . , y_{0}^{(n-1)} $ (даже в малой области). \\
\indent
Тогда мы скажем, что \textit{n постоянных} $ C_1$, $C_2$, . . . , $ C_n $, в выражении (14') или (15) \textit{не являются существенными}, и эти выражения не представляют общего решения. \\
\indent
П\,р\,и\,м\,е\,ч\,а\,н\,и\,е\, 2. Как и в уравнениях первого порядка, может представиться случай, когда формула вида (14'), содержащая n произвольных постоянных, не дает всех частных решений, определяемых начальными данными Коши.

\textit{Пример 1}. Уравнение $y (1-\ln y) y^n + (1+\ln y) y'^2 = 0$ при $y \neq 0$, $y \neq e$, приводится к виду (1') с правой частью непрерывною и имеющею непрерывные производные по $y$ и $y'$. Следовательно, решение, определяемое начальными данными $x_{0}, y_{0}, y'_{0}$ при условии $ y_{0} \neq 0, \neq e$, является обыкновенным. Решение, содержащее две произвольных постоянных $a$, $b$, дается (как легко проверить) формулой  $ \ln y = \frac{x+a}{x+b}$. Однако из этого решения нельзя получить частных решений, определяемых начальными условиями: $ x=x_{0}$, $y=y_{0}( \neq 0, \neq e)$, $y_{0}'=0$. Эти частные решения получаются из формулы: $y=C$ (легко видеть, что постоянное значение $y$ удовлетворяет уравнению). В этом случае мы принуждены сказать, что общее решение уравнения дается двумя формулами: $ \ln y = \frac{x+a}{x+b}$, $y=C$. \\
П\,р\,и\,м\,е\,ч\,а\,н\,и\,е\, 3. Уравнение вида (1) может быть разрешено относительно $y^{(n)}$, т. е. приведено к виду (1') вблизи любых начальных значений $x_{0}, y_{0}, y_{0}', . . . , y_{0}^{(n)}$, удовлетворяющих условию
$$ F(x_{0}, y_{0}, y_{0}', . . . , y_{0}^{(n)}) = 0, $$
если только для этих значений аргументов производная $ \frac{\partial F}{\partial y^{(n)}} \neq 0$. Все последующие рассуждения имеют силу только в этом предположении. Рассмотрение тех значений, для которых производная $ \frac{\partial F}{\partial y^{(n)}}$ обращается в нуль, привело бы к рассмотрению особых решений уравнения n-го порядка; мы на этой теории останавливаться не будем. \\
\indent
Дальнейшей целью настоящей главы будет -- установить некоторые случаи, когда уравнение (1) или (1') может быть проинтегрировано до конца в квадратурах или, по крайней мере, когда задача его интегрирования может быть сведена к интегрированию диференциального уравнения порядка меньшего, чем \textit{n}.
\end{document}
